\DocumentMetadata
  {
    lang=en-US,
    pdfversion=2.0,
    pdfstandard=ua-2,
    tagging=on
  }
\documentclass{article}

\usepackage{newverbs}
\usepackage{xcolor}

\title{newverbs tagging test}

\newverbcommand\cverb{\color{blue}}{}
\MakeSpecialShortVerb{\qverb} {\"}
\MakeSpecialShortVerb{\fverb*}{\|}
\Verbdef*\foo+%& $^_+   \verbdef*\baz+%& $^_+ 

\newenvironment{myenv}{\collectverbenv{\textit}}{}

\begin{document}

\tableofcontents

By default the package provides \qverb=\qverb= and
\fverb=\fverb= ready for use. We have added \cverb=\cverb=.

Now code like |pdflatex myfile| can be "enter"ed in a
simplified way. "\DeleteShortVerb" removes previously
defined shorthands\DeleteShortVerb\| again: |see|?

\section{Difficult chars (\foo\ or \baz)}
This hides \verb*+%& $^_+ within \verb+\foo+
and \verb+\baz+ so that it can be used in
the heading. Warning: \verb*+%& $^_+ is
equal to \baz{} but not to \foo{}!

\begin{myenv}
bla bla

#$^&*_
\end{myenv}

\end{document}
