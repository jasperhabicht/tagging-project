% taken from delimset doc
\DocumentMetadata
  {
    lang=en-US,
    pdfversion=2.0,
    pdfstandard=ua-2,
    tagging=on
  }
\documentclass[12pt]{article}

\usepackage[margin=2cm]{geometry}
\usepackage{amsfonts}
\UseName{sys_if_engine_opentype:TF}
  {\usepackage{unicode-math}}
  {\usepackage{amsmath}}
\usepackage{delimset}

\title{delimset tagging test}

\begin{document}

sizes for default brackets:
\[
\brk^0{x},\quad
\brk^1{x},\quad
\brk^2{x},\quad
\brk^3{x},\quad
\brk^4{x}
\]

styles for default brackets:
\[
\brk[r]{x},\quad
\brk[s]{x},\quad
\brk[c]{x},\quad
\brk[a]{x}
\]

nested brackets:
\[
\brk[c]^2{\brk[s]!{\brk{ax+b}x+c}x+d}
\]

default absolute value, norm and default evaluations:
\[
\abs*{\frac{ax+b}{cx+d}},\qquad
\norm*{\frac{ax+b}{cx+d}},\qquad
\eval*{\frac{ax+b}{cx+d}}_{x=0},\qquad
\eval[s]*{\frac{ax+b}{cx+d}}_{x=0}^{x=\infty}
\]

outer delimiter spacing:
\begin{align*}
&\square\brk^0{x}\square,&&\square\brk^1{A^k}\square,
\\
&\square\brk*{x}\square,&&\square\brk*{A^k}\square
\end{align*}

delimiter sizes in exponents:
\[
e^{\brk{ax+b}},\qquad
e^{\brk!{ax+b}}
\]

delimiter declaration:
\DeclareMathDelimiterSet{\braket}[2]
  {\selectdeliml<#1\selectdelim|#2\selectdelimr>}
\[
\braket!{\psi}{\psi},
\quad
\braket*{\psi}{\psi\big.}
\]

delimiter usage:
\[
\delimpair<|>!{\psi}{\psi}
\]

conditional set, alternative layouts:
\[
\delimpair\{{[m]|}\}!{2n}{n\in\mathbb{Z}},
\quad
\delimpair\{{[b]|}\}!{2n}{n\in\mathbb{Z}},
\quad
\delimpair\{{[i]|}\}!{2n}{n\in\mathbb{Z}},
\quad
\delimpair\{{[p]|}\}!{2n}{n\in\mathbb{Z}},
\quad
\delimpair\{|\}!{2n}{n\in\mathbb{Z}},
\quad
\delimpair\{{*;}\}!{2n}{n\in\mathbb{Z}}
\]
conditional set, alternative layouts with variable size:
\[
\delimpair\{{[m]|}\}*{2n}{n\in\mathbb{Z}\big.},
\quad
\delimpair\{{[b]|}\}*{2n}{n\in\mathbb{Z}\big.},
\quad
\delimpair\{{[i]|}\}*{2n}{n\in\mathbb{Z}\big.},
\quad
\delimpair\{{[p]|}\}*{2n}{n\in\mathbb{Z}\big.},
\quad
\delimpair\{|\}*{2n}{n\in\mathbb{Z}\big.},
\quad
\delimpair\{{*;}\}*{2n}{n\in\mathbb{Z}\big.}
\]

delimiter definition:
\newcommand{\comm}{\delimpair[{*,}]}
\[
\comm!{\comm{A}{B}}{C}
+\comm!{\comm{B}{C}}{A}
+\comm!{\comm{C}{A}}{B}
=0
\]

alternative representation:
\renewcommand{\comm}{\delimpair[{*;}]}
\[
\comm!{\comm{A}{B}}{C}
+\comm!{\comm{B}{C}}{A}
+\comm!{\comm{C}{A}}{B}
=0
\]

display individual delimiters of a set:
\renewcommand{\braket}{\delimpair<|>}
\[
\braket{A}{B}
\to \braket( A \braket| B \braket),
\quad
\braket*( A\big. \braket*| B_{} \braket*),
\quad
\braket^1( A \braket^3| B \braket^2)
\]

placing indices before a bracket
(does not work in variable-size mode because
the final size is not available for the enclosed expressions):
\DeclareMathDelimiterSet{\quadindex}[5]
  {\selectdeliml.^{#2}_{#3}\mathord{}\selectdelim[o][
  {#1}\selectdelim[o]]^{#4}_{#5}\selectdelimr.}
\[
\quadindex^2{\frac{x}{y}}{1}{2}{3}{4}
\]

\end{document}