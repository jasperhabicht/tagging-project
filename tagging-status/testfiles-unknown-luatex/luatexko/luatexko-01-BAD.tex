\DocumentMetadata
  {
    lang=ko,
    pdfversion=2.0,
    pdfstandard=ua-2,
    tagging=on
  }
\documentclass{article}

\usepackage{unicode-math}
\usepackage{luatexko}

\defaultfontfeatures+{Renderer=HarfBuzz}

% https://fonts.google.com/noto/specimen/Noto+Serif+KR
\setmainhangulfont{Noto Serif KR}[
  Script=Hangul,
  Language=Korean,
  ]
% https://fonts.google.com/noto/specimen/Noto+Sans+KR
\setmathhangulfont{Noto Sans KR}[
  Script=Hangul,
  Language=Korean,
  ]
\newhangulfontfamily\verticalhangulfont{Noto Sans KR}[
  Renderer=Node,
  Script=Hangul,
  Language=Korean,
  UprightFont=* Light,
  BoldFont=* Bold,
  InterLatinCJK=.125em,
  Vertical=Alternates,
  RawFeature=vertical,
  CompressPunctuations,
  InterCharStretch=1pt,
  CharRaise=1pt,
  CharacterWidth=Full,
]

\title{luatexko tagging test}

\begin{document}
패키지 옵션

$가^{나^다}=\sum_i^\infty a_i$

\dotemph{드러냄표}

\ruby{漢字}{한자}

\xxruby{漢字}{한자}

\uline{밑줄을 그을 수 있다}

\sout{취소선을 그을 수 있다}

\uuline{밑줄을 두 줄 긋는다}

\xout{빗금으로 취소할 수 있다}

\uwave{물결표로 밑줄을 삼는다}

\dashuline{대시로 밑줄을 삼는다}

\dotuline{밑줄을 점선으로 긋는다}

\hellipsis

\begin{vertical}{20em}
\verticalhangulfont
世솅〮宗조ᇰ御ᅌᅥᆼ〮製졩〮訓훈〮民민正져ᇰ〮音ᅙᅳᆷ
\end{vertical}

\begin{verticaltypesetting}
\verticalhangulfont
世솅〮宗조ᇰ御ᅌᅥᆼ〮製졩〮訓훈〮民민正져ᇰ〮音ᅙᅳᆷ
\begin{horizontal}{20em}
\normalfont
世솅〮宗조ᇰ御ᅌᅥᆼ〮製졩〮訓훈〮民민正져ᇰ〮音ᅙᅳᆷ
\end{horizontal}
\end{verticaltypesetting}

\end{document}