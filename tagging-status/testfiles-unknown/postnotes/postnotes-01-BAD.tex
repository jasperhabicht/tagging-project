\ExplSyntaxOn
\sys_gset_rand_seed:n{1000}
\ExplSyntaxOff
\DocumentMetadata{
  tagging=on,
  pdfversion=2.0,
  pdfstandard=ua-2,
  % alternatively:
  % pdfversion=1.7,
  % pdfstandard=ua-1,
  lang=en,
}

\documentclass{article}

% This file focuses on the document elements which are actually being tagged
% by the support provided by the package: the notes themselves, the marks,
% cross-references (both LaTeX and PDF ones), and some special cases.
\usepackage{postnotes}
\postnotesetup{multiple}

\usepackage{zref-user,zref-hyperref}
\usepackage{hyperref}
\hypersetup{hidelinks}

\title{Example file for postnotes PDF tagging support}

\begin{document}

\section{Section 1}

\postnote[label=en:mark:1,zlabel=zen:mark:1]{%
  \label{en:text:1}\zlabel{zen:text:1}1}

\postnoteref{en:text:1}\par
\postnotezref{zen:text:1}\par
\postnoteref{en:mark:1}\par
\postnotezref{zen:mark:1}

\postnote[label=en:mark:2,zlabel=zen:mark:2]{%
  \label{en:text:2}\zlabel{zen:text:2}2}

\postnoteref{en:text:2}\par
\postnotezref{zen:text:2}\par
\postnoteref{en:mark:2}\par
\postnotezref{zen:mark:2}

% A nomark note is tagged as a NonStruct element, but gets all references.
\postnote[nomark]{\label{en:text:3}3}
\postnoteref{en:text:3}

% The multisep is tagged as an artifact.
\postnote{4}\postnote{5}

\printpostnotes

\end{document}
