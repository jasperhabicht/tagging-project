\DocumentMetadata
  {
    lang=en-US,
    pdfversion=2.0,
    pdfstandard=ua-2,
    tagging=on,
  }
\documentclass{article}
\usepackage{layouts}
\usepackage{kantlipsum}

\title{layouts tagging test}

\begin{document}
\newlength{\pwlayi}
\setlength{\pwlayi}{0.4375\textwidth}
\newlength{\pwlayii}
\setlength{\pwlayii}{0.5\pwlayi}
\begin{figure}
\centering
\begin{minipage}[b]{\pwlayi}
\drawaspread{\pwlayii}{1.294}{1.618}{0.176}{1.037}{1.685}{0}
\end{minipage}
\hfill
\begin{minipage}[b]{\pwlayi}
\drawaspread{\pwlayii}{1.5}{1.5}{0.111}{1.5}{2.0}{0}
\end{minipage}
\caption{(Left) The \LaTeX{} book spread;
(Right) Spread for many of Gutenberg’s books}
\label{fig:spread}
\end{figure}

\begin{figure}
\oddpagelayoutfalse
\twocolumnlayouttrue
\pagediagram
\caption{Left-hand two-column page layout parameters} \label{fig:pplt}
\end{figure}

\begin{figure}
\currentpage
\oddpagelayouttrue
\pagedesign
\caption{Odd page layout for this document} \label{fig:ptrs}
\end{figure}

\begin{figure}
\currentpage
\trypaperwidth{11in}
\trypaperheight{8.5in}
\trytextwidth{500pt}
\trycolumnsep{40pt}
\trycolumnseprule{3pt}
\tryhoffset{-0.5in}
\tryvoffset{0.5in}
\printheadingsfalse
\drawdimensionstrue
\twocolumnlayouttrue
\pagedesign
\caption{An experimental page layout}\label{fig:pudf}
\end{figure}

\pagevalues

\begin{figure}
\paragraphdiagram
\caption{Paragraph parameters}\label{fig:fpara}
\end{figure}

\kant[3]
\currentparagraph
\begin{figure}
\paragraphdesign
\caption{Paragraphs in this document}\label{fig:dpara}
\end{figure}

\begin{figure}
\floatpagediagram
\caption{Float and text page parameters}\label{fig:fpp}
\end{figure}

\begin{figure}
\currentfloatpage
\trytotalnumber{3}
\trytopnumber{2}
\trytopfraction{0.7}
\trytextfraction{0.2}
\trybottomfraction{0.3}
\trybottomnumber{1}
\setlayoutscale{0.25}
\floatpagedesign
\caption{The standard \LaTeX{} float and text page settings}
\label{fig:fpstd}
\end{figure}

\begin{figure}
\setlayoutscale{0.9}
\floatdiagram
\caption{Float parameters}\label{fig:flp}
\end{figure}

\begin{figure}
\currentfloat
\tryintextsep{\intextsep}
\trytopfigrule{0.5pt}
\trybotfigrule{1pt}
\setlayoutscale{0.9}
\floatdesign
\caption{Float layout with rules}\label{fig:fludf}
\end{figure}

\begin{figure}
\listdiagram
\caption{List parameters} \label{fig:lstp}
\end{figure}

\begin{enumerate}
\item Figure~\ref{fig:lstenum} illustrates the layout of an
\texttt{enumerate} list.
\currentlist
\begin{figure}
\listdesign
\caption{Layout of an \texttt{enumerate} list} \label{fig:lstenum}
\end{figure}
\end{enumerate}

\begin{figure}
\setlayoutscale{1}
\headingdiagram{ }
\caption{Display heading parameters}\label{fig:hdp}
\end{figure}
\begin{figure}
\setlayoutscale{1}
\runinheadtrue
\headingdiagram{ }
\caption{Run-in heading parameters}\label{fig:hrp}
\end{figure}

text\footnote{a footnote}

\begin{figure}
\setlayoutscale{0.4}
\setlabelfont{\huge\itshape}
\footnotediagram
\caption{The footnote parameter layout} \label{fig:fp}
\end{figure}

\begin{figure}
\currentfootnote
\setlayoutscale{0.4}
\footnotedesign
\caption{The current footnote layout}\label{fig:ftry}
\end{figure}

\begin{figure}
\setlayoutscale{0.8}
\tocdiagram
\caption{Table of Contents entry parameters}\label{fig:tocp}
\end{figure}

\begin{figure}
\setlayoutscale{0.8}
\currenttoc
\tocdesign
\caption{Typical Table of Contents entry for this document} \label{fig:thistoc}
\end{figure}

\drawfontframe{\Huge\textbf{g}}

\drawfontframelabel{\Huge Q}

\end{document}