\DocumentMetadata
  {
    lang=en-US,
    pdfversion=2.0,
    pdfstandard=ua-2,
    tagging=on
  }
\documentclass{article}
\usepackage{tabulary}

\title{tabulary tagging test}

\begin{document}

\begin{center}

With C columns
\begin{tabulary}{\linewidth}{C|C@{ (an @ expr.) }C}
1&the rain in spain falls mainly on the plain&
the rain in spain falls mainly on the plain
the rain in spain falls mainly on the plain\\
a&b&c\\
a a&b b&c c
\end{tabulary}

\bigskip

With J columns
\begin{tabulary}{\linewidth}{J|J@{ (an @ expr.) }J}
1&the rain in spain falls mainly on the plain&
the rain in spain falls mainly on the plain
the rain in spain falls mainly on the plain\\
a&b&c\\
a a&b b&c c
\end{tabulary}

\bigskip

With L, R and C columns, and a \verb|\multicolumn|
\begin{tabulary}{\linewidth}{LR|LC}
1&the rain in spain falls mainly on the plain&
the rain in spain falls mainly on the plain
the rain in spain falls mainly on the plain&
and now for something completely different\\
x&\multicolumn{3}{c}
  {some multicolumn text across columns 2--4}\\
a&b&c&d\\
a a&b b&c c&d d
\end{tabulary}
\end{center}

\end{document}