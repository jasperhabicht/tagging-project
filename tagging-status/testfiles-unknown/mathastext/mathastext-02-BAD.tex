% sample taken from mathastext.dtx
\DocumentMetadata
  {
    lang=en-US,
    pdfversion=2.0,
    pdfstandard=ua-2,
    tagging=on
  }
\documentclass{article}

\usepackage[no-math]{fontspec}
\setmainfont[Ligatures=TeX]{Libertinus Serif}
\usepackage[defaultmathsizes,LGRgreek]{mathastext}
\MTgreekfont{LibertinusSerif-TLF}
\Mathastext

\title{mathastext tagging test - luatex/xetex}

\begin{document}

Let $(X,Y)$ be two functions of a variable $a$. If they obey  the differential
system  $(VI_{\nu,n})$: 
\begin{align*}
a\frac{d}{da} X &= \nu
X - (1 - X^2)\frac{2n a}{1 - a^2}\frac{aX+Y}{1+a XY} \\  
a\frac{d}{da} Y &= -(\nu+1) Y
+ (1 - Y^2)\frac{2n a}{1 - a^2}\frac{X+aY}{1+a XY} 
\end{align*}
then the quantity $q = a \frac{aX+Y}{X+aY}$
satisfies as function of $b= a^2$  the $P_{VI}$ differential equation:
\begin{equation*}
\begin{split}
\frac{d^2 q}{db^2} = \frac12\left\{\frac1q+\frac1{q-1}
+\frac1{q-b}\right\}\left(\frac{dq}{db}\right)^2 - \left\{\frac1b+\frac1{b-1}
+\frac1{q-b}\right\}\frac{dq}{db}\\+\frac{q(q-1)(q-b)}{b^2(b-1)^2}\left\{\alpha+\frac{\beta
b}{q^2} + \frac{\gamma (b-1)}{(q-1)^2}+\frac{\delta
b(b-1)}{(q-b)^2}\right\}
\end{split}
\end{equation*}
with
parameters
$(\alpha,\beta,\gamma,\delta) = (\frac{(\nu+n)^2}2,
\frac{-(\nu+n+1)^2}2, \frac{n^2}2, \frac{1 - n^2}2)$.

Test of uppercase Greek in math: $\Alpha\Beta\Gamma\Delta\Xi\Omega$.

\end{document}