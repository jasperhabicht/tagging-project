% adapted from test-luamplib-latex.tex
\DocumentMetadata
  {
    lang=en-US,
    pdfversion=2.0,
    pdfstandard=ua-2,
    tagging=on
  }
\documentclass{article}
\usepackage{unicode-math}
\setmainfont{Latin Modern Roman}
\usepackage{luamplib}
\usepackage{graphicx,xcolor}
\everymplib{ beginfig(0); }
\everyendmplib{ endfig; }
\mpliblegacybehavior{true}%

\title{luamplib tagging test - 2}

\begin{document}
A%
\begin{mplibcode}[alt=example 1]
%% test all printable ascii chars in comments
%%    (  2  <  F  P  Z  d   n   x
%%    )  3  =  G  Q  [  e   o   y
%%    *  4  >  H  R  \  f   p   z
%% !  +  5  ?  I  S  ]  g   q   {
%% "  ,  6  @  J  T  ^  h   r   |
%% #  -  7  A  K  U  _  i   s   }
%% $  .  8  B  L  V  `  j   t   ~
%% %  /  9  C  M  W  a  k   u  DEL
%% &  0  :  D  N  X  b  l   v
%% ´  1  ;  E  O  Y  c  m   w
	fill fullcircle scaled 20; %% actual <tab> to make sure it works
\end{mplibcode}%
B\par
\everymplib{}\everyendmplib{}% reset toks
A%
\begin{mplibcode}[alt=example 2]
verbatimtex \lower.2em etex
beginfig(0);
draw origin--(1cm,0) withcolor \mpcolor{teal};
draw btex g etex withcolor \mpcolor{orange};
draw btex\tracingcommands0
  multi%
  \begin{huge}line\end{huge}
  texcode
  with \%
  etex shifted (10,0);
draw bbox currentpicture;
endfig;

beginfig(18);
numeric u;
u = 1cm;
draw (0,2u)--(0,0)--(4u,0);
pickup pencircle scaled 1pt;
draw (0,0){up}
  for i=1 upto 8: ..(i/2,sqrt(i/2))*u  endfor;
label.lrt(btex $\sqrt x$ etex, (3,sqrt 3)*u);
label.bot(btex $x$ etex, (2u,0));
label.lft(btex $y$ etex, (0,u));
endfig;
\end{mplibcode}%
B\par
A%
\begin{mplibcode}[alt=example 3]
beginfig(2);
numeric u; u=1cm;
z1=-z2=(-u,0);
for i = 1 upto 3:
  draw z1..(0, i*u)..z2;
  label.top(TEX("$e_{" & decimal(i) & "}$"), (0, i*u))
  withcolor \mpcolor{red!70!blue};
endfor;
endfig;
\end{mplibcode}%
B\par\mplibforcehmode
\mplibcodeinherit{enable}%
\mplibglobaltextext{enable}%
\everymplib{ beginfig(0);}\everyendmplib{ endfig;}%
\begin{mplibcode}[alt=example 4]
 label(btex $\sqrt{2}$ etex, origin);
 draw fullcircle scaled 20;
 picture pic; pic := currentpicture;
\end{mplibcode}%
\begin{mplibcode}[alt=another example]
 currentpicture := pic scaled 2;
\end{mplibcode}%
\mplibnoforcehmode
\everymplib{}\everyendmplib{}%
\mplibcodeinherit{disable}%
\mplibglobaltextext{disable}\par
\mplibsetformat{metafun}%
\begin{mplibcode}[alt=example 5]
message "Hello World!";
verbatimtex \moveright 0.4\hsize etex
beginfig(0);
path p; p:= fullcircle scaled 2cm yshifted .5cm;
fill p withcolor transparent("normal", 0.5, red);
fill p rotated 120 withcolor transparent("normal", 0.5, green);
fill p rotated 240 withcolor transparent("normal", 0.5, blue);
endfig;
verbatimtex \leavevmode etex
picture p; p := btex MetaPost etex scaled 2;
beginfig(1);
linear_shade(bbox p,0,blue,.7white);
draw p withcolor white;
endfig;
verbatimtex \kern10pt etex
beginfig(2);
circular_shade(bbox p,0,blue,.7white);
draw p withcolor white;
endfig;
\end{mplibcode}%

\newbox\mympbox
\def\prependtomplibbox{\global\setbox\mympbox}%
\begin{mplibcode}[alt=example 6]
%verbatimtex \global\setbox\mympbox etex
beginfig(0);
breadth=.667\mpdim\linewidth;
height=2pt;
x1=0; x2=x6=.333x4; x5=x3=.667x4;
x4=breadth;
y1=y4=height/2; y2=y3=height; y5=y6=0;
fill z1--z2--z3--z4--z5--z6--cycle;
endfig;
\end{mplibcode}%

\mplibnoforcehmode
\mplibnumbersystem{double}%
\begin{mplibcode}[alt=example 7]
beginfig(0);
u := 10**5*(10**-4);
draw unitsquare scaled u;
endfig;
\end{mplibcode}%
\mplibsetformat{plain}%
\begin{mplibcode}[alt=example 8]
  input graph;
  beginfig(0);
  draw begingraph(100,100);
    gdraw (10,10)--(30,35)--(50,25)--(70,80)--(90,90);
    autogrid(otick.bot,);
    for y=20,40,60,80:
      grid.lft(format("%e",1000y), y) withcolor .85white;
    endfor
    endgraph;
  endfig;
\end{mplibcode}%
\mplibtextextlabel{enable}%
\begin{mplibcode}[alt=example 9]
beginfig(0);
dotlabel.rt("$\sqrt2$",origin);
endfig;
\end{mplibcode}%
\leavevmode
\begin{mplibcode}[alt=example 10]
   D := sqrt(2)**7;
   beginfig(0);
   draw fullcircle scaled D;
   VerbatimTeX("\gdef\Dia{" & decimal D & "}");
   endfig;
\end{mplibcode}%
diameter:\Dia bp.%
\begin{mplibcode}[alt=example 11]
  vardef rotatedlabel@#(expr str, loc, angl) =
    draw thelabel@#(str, loc) rotatedaround(loc, angl)
  enddef;

  beginfig(1);
    rotatedlabel.top(textext("Rotated!"), origin, 45);
  endfig;
\end{mplibcode}%
\par
\begin{mplibcode}[alt=example 12]
beginfig(1)
  draw mplibgraphictext "\bfseries Funny$\sqrt{2}$"
  fakebold 2 % fontspec option
  drawcolor blue fillcolor "red!50" % l3color expression
  scaled 3 rotated 30 ;
picture p;
p:=mplibgraphictext "\bfseries\itshape xyz";
draw p scaled 3 shifted (40,0);
endfig;
\end{mplibcode}%
\par
\mplibsetformat{metafun}%
\begin{mplibcode}[alt=example 13]
beginfig(1)
fill unitsquare xyscaled (\mpdim\textwidth,1cm)
    withshademethod "linear"
    withshadevector (0,1)
    withshadestep (
       withshadefraction .5
       withshadecolors (red,"blue!50")
    )
    withshadestep (
       withshadefraction 1
       withshadecolors ("blue!50",green)
    )
    ;
endfig;
\end{mplibcode}%
\leavevmode
\mpfig[alt=fig example 1]
picture Q, u, e;
Q := mplibglyph "Q" of "texgyrepagella-bolditalic.otf" scaled .1;
u := mplibglyph "u" of "texgyrepagella-bolditalic.otf" scaled .1 shifted lrcorner Q;
e := mplibglyph "e" of "texgyrepagella-bolditalic.otf" scaled .1 shifted lrcorner u;
i:=0;
totallen := length Q + length u + length e;
for pic=Q, u, e:
  for item within pic:
    i:=i+1;
    fill pathpart item
    if i < totallen: withpostscript "collect"; fi
  endfor
endfor
  withshademethod "linear"
  withshadedirection (0.5,2.5)
  withshadecolors (.7red,.7yellow)
  ;
\endmpfig
\everymplib[@mpfig]{ drawoptions(withcolor mplibrgbtexcolor "olive"); }%
\mpfig*[alt=fig example 2] input boxes \endmpfig
\mpfig[alt=fig example 3] circleit.a(btex\tracingcommands0 Box 1 etex); drawboxed(a); \endmpfig
\def\mpfiginstancename{mympfig}%
\mpfig[alt=fig example 4]
draw mpliboutlinetext.b ("$\displaystyle\frac{1}{1-x^2}$")
    (withcolor .6[red,white])
    (withpen pencircle scaled .2 withcolor red)
    scaled 4 ;
\endmpfig
\par
\mpfig[alt=fig example 5]
  draw mpliboutlinetext.r
    ("Question")
    ( withpen pencircle scaled .3 )
    (
      withshademethod "linear"
      withshadedirection (2.5, 0.5)
      withshadecolors(red, blue)
    )
    scaled 4;
\endmpfig
\par\leavevmode
\mppattern{mypatt}
[
  xstep = 5, ystep = 6,
  matrix = "rotated 90 scaled .75",
]
\mpfig[alt=fig example 6]
draw (origin--right+up) scaled 5 withcolor 1/3[blue,white] ;
draw (up--right) scaled 5 withcolor 1/3[red,white] ;
\endmpfig
\endmppattern
\mpfig[alt=fig example 7]
draw unitsquare shifted -center unitsquare scaled 45 withpostscript "collect" ;
draw fullcircle scaled 100 withpattern "mypatt" withpen pencircle scaled 1
  withcolor \mpcolor{red!50!blue!50} withpostscript "evenodd" ;
\endmpfig
\begin{mppattern}{pattuncolored}
  [
    colored = false,
    matrix = "slanted .3 rotated 30",
  ]
  \tiny\TeX
\end{mppattern}\relax
\mpfig[alt=fig example 8]
  picture tex;
  tex = mpliboutlinetext.p ("\bfseries \TeX");
  for i=1 upto mpliboutlinenum:
    draw image(
    j:=0;
    for item within mpliboutlinepic[i]:
      j:=j+1;
      draw pathpart item scaled 10
      if j < length mpliboutlinepic[i]:
          withpostscript "collect"
      else:
          withpattern "pattuncolored"
          withpen pencircle scaled 1/2
          withcolor (i/4)[red,blue]
      fi;
    endfor
    )
    withfademethod "linear"
    ;
  endfor
\endmpfig
\par\leavevmode
\mpfig[alt=fig example 9]
  picture mill; mill = btex \includegraphics[width=100bp]{mill} etex;
  draw mill;
  mill := mill shifted 125right;
  draw mill
    withfademethod "circular"
    withfadecenter (center mill, center mill)
    withfaderadius (30, abs(lrcorner mill - llcorner mill)/2)
    withfadeopacity (1, 0)
    ;
\endmpfig
\par
\def\test#1{%
  \mpfig[alt=fig example 10]
  fill unitsquare shifted -center unitsquare scaled 200
    withshademethod "linear"
    withshadecolors (.3[red,white], .3[blue,white])
    ;
  label( btex MetaPost etex scaled 3, origin) ;
  draw image(
    fill fullcircle scaled 100 shifted 25left
      withcolor .8white
      withtransparency (2,1)
      ;
    fill fullcircle scaled 100 shifted 25right
      withcolor .8white
      withtransparency (2,1)
      ;
  )
  asgroup "#1"
  withgroupname "test:#1"
  withtransparency (1,.5)
  ;
  currentpicture := currentpicture scaled 2/3;
  \endmpfig
}%
\hbox{\test{isolated,knockout}\,\test{isolated}\vbox to\MPheight{\vss\hbox{~isolated}\vss}}%
\hbox{\test{knockout}\,\test{}}%
\hbox to\MPwidth{\hss knockout\hss}%
\leavevmode
\mpfig[alt=fig example 11] usemplibgroup "test:isolated,knockout" scaled 2/3 rotated 15 ; \endmpfig
\usemplibgroup[artifact]{test:isolated,knockout}%
\begin{mplibgroup}{mytex}[matrix="rotated 15"] \TeX \end{mplibgroup}%
\hbox to0pt{\hss\vrule width.5pt height5pt depth5pt\hss}%
\hbox to0pt{\hss\vrule width10pt height.25pt depth.25pt\hss}%
\usemplibgroup[artifact]{mytex}%
\mpfig[alt=fig example 12] usemplibgroup "mytex"; draw (left--right) scaled 5; draw (up--down) scaled 5; \endmpfig
\par
\mpfig[alt=fig example 13]
  picture test; test = mplibgraphictext "\textbf{MPLIB}"
    fakebold 1 fillcolor "red!70" drawcolor .7red scaled 7;
  draw test withpattern "mypatt" ;
\endmpfig
\tracingcommands0

\vskip 2\baselineskip
\mplibcodeinherit{disable}%

\everymplib[instanceOne]{beginfig(1);}
\everyendmplib[instanceOne]{endfig;}

\begin{mplibcode}[instanceOne,alt=example 14]
  picture TeX;
  TeX := btex \TeX etex;
a := 1cm;
draw fullcircle scaled a;
draw btex a circle with $d=a$ etex shifted (a,0);
draw TeX;
\end{mplibcode}%
Current instance name is: \currentmpinstancename \vskip 2\baselineskip

\begin{mplibcode}[instanceTwo,alt=example 15]
beginfig(1);
if not known a:
  draw btex code is not inherited from an instance with a different name etex;
else:
  errmessage("Variable was inherited from a different instance");
fi;
endfig;
\end{mplibcode}%
Current instance name is: \currentmpinstancename \vskip 2\baselineskip

\begin{mplibcode}[alt=example 16]
beginfig(1);
if not known a:
  draw btex code is not inherited if instance name is not listed etex;
else:
  errmessage("Variable was inherited from a different instance");
fi;
a := 1cm;
endfig;
\end{mplibcode}%
Current instance name is: \currentmpinstancename (should be empty) \vskip 2\baselineskip

\begin{mplibcode}[alt=example 17]
beginfig(1);
if not known a:
  draw btex code is not inherited if mplibcodeinherit is disabled and instance name is not explicitly set etex;
else:
  errmessage("Variable was inherited when code inheritance is turned off and instance name is not set");
fi;
endfig;
\end{mplibcode}%
Current instance name is: \currentmpinstancename (should be empty) \vskip 2\baselineskip

\begin{mplibcode}[instanceOne,alt=example 18]
draw unitsquare scaled a;
draw btex a square with side $=a$, inherited from the same instance etex shifted (3/2a, 1/2a);
  draw TeX;
\end{mplibcode}%
Current instance name is: \currentmpinstancename \vskip 2\baselineskip

\end{document}
